\chapter{Conclusion} \label{cha:Concl}
This thesis has contributed to the ongoing research of the Department of Mechanical Engineering of the KU Leuven on Wheelchair Navigation Assistance. The goal of this work was to expand the current LPP, the circular LPT, with a more flexible curve, enabling it to plan more complex paths to a desired destination and therefore providing better navigation assistance, as the collision-free paths are used to model the user's intention along with the path the sAMR will execute.

The developed clothoidal LPT consists of a set of kinematically feasible trajectories based on clothoids (a curve whose curvature changes linearly with its arc length) starting from the current pose of the robot and connecting it with a set of discrete end-poses in the surrounding of the robot. With this fixed set of trajectories, a lookup table is built in order to efficiently obtain accurate collision-free paths by adapting the length of each trajectory.

Several simulations were designed to evaluate the improved planning performances of the clothoidal LPT as compared to its predecessor, the circular LPT. These experiments confirmed that by adopting a more flexible curve geometry, the clothoidal LPT’s ability to plan a passage through a narrow opening was less dependent on the current pose of the sAMR as its predecessor.  This has increased the amount of paths in the clothoidal LPT by six which has led in turn to an increase in overall executing time by four. The main improvement that could be made to the current design of the clothoidal LPT in terms of computing time would be to change the current locations of the EPs, which are the main reason behind the large increase of paths, and decide on their location more accurately.

The focus of the developed LPPA has been on curve geometry and path length adaptation for static obstacles. A conceptual solution using an efficient search space (the distance-time collision space) has been put forward to cover also for the prevention of collision in a dynamic environment. Building on the respective strengths of the LPT and optimal motion generation, this solution provides an optimal speed profile resulting in a collision-free motion for a fixed path in an environment with dynamic obstacles. This could be extended by using a grid search algorithm with re-planning abilities, since there is a similarity between the occupied space caused by the dynamic obstacles between similar paths, resulting in faster calculation time.

The need for a cooperative planner, which anticipates the cooperation of people in dense crowds has been presented as a possible solution to the FRP and would therefore enable the sAMR to navigate efficiently and safely in densely populated environments. Social compliant robotic navigation would be achieved by integrating a dynamic social cost map accounting for discomfort caused to interacting persons by the wheelchair as it moves through crowds.